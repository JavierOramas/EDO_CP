\documentclass{article}
\title{CP3 Ecuaciones exactas y reducibles a exactas}
\author{Javier Alejandro Oramas López C212}

\begin{document}
    
    \section{Resuelva las siguientes ecuaciones diferenciales:}
        \subsection{$(1 + ye^{xy})dx + (2y + xe^{xy})dy = 0$}
            Verificamos si es una ecuación diferencial exacta:
            \[ \frac{d(1+ye^{xy})}{dy} = e^{xy} + xye^{xy}\]
            \[ \frac{d(2y+xe^{xy})}{dx} = e^{xy} + xye^{xy}\]
            Calculamos la INtegral

            \[\int 1+ye^{xy} dx = x+e^{xy} + g(y) \]
            \[\frac{d(x+e^{xy}+g(y))}{dy} = xe^{xy}+g^{'}(y)\]

            igualamos a M
            \[xe^{xy}+g^{'}(y) = xe^{xy} +2y\]
            \[g^{'}(y) = 2y\]
            \[g(x) = y^2\]

            Sustituimos en el resultado de la integral e igualamos a la constante
            \[x+xe^{xy}+y^2 = C\]
        \subsection{$\frac{ydx-xdy}{y^2}+xdx = 0$}
            Multiplicamos por $y^2$ en ambos miebros
            \[ydx-xdy+y^2xdx = 0\]
            \[(y+y^2x)dx -xdy = 0\]

            Hallando las derivadas parciales
            \[M_y = 1+2xy\]
            \[N_x = -1\]

            Como no son iguales tenemos que hallar un factor de integración para convertir la ecuación en exacta
            \[\frac{M_y-N_x}{N} = \frac{1+2xy+1}{-x} = \frac{2+2yx}{-x}\]
            \[\frac{N_x-M_y}{M} = \frac{-1-1-2yx}{y+y^2x} = \frac{-2-2yx}{y+y^2x}\]

            Como no quedan funciónes en función de x o y respectivamente tenemos que recurrir al metodo
            \[M_y-N_x= m \frac{N}{x} - n \frac{M}{y}\]

            \[1+2xy-1 = m \frac{-x}{x} - n \frac{y_y^2x}{y}\]
            \[2xy = -m -n (1+xy)\]
            \[2xy = -m-n-nxy\]


            \[n = -2\]
            \[m = -n = 2\]

            $x^2+y^-2$ Es el factor de integración 

\end{document}