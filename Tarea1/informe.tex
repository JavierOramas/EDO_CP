\documentclass{article}
\usepackage[utf8]{inputenc}
\usepackage{parskip}
\title{Ecuaciones diferenciales ordinarias. Tarea I}
\author{
    Kevin Talavera Díaz C211\\
    Lía Zerquera Ferrer C212\\
    Javier Alejandro Oramas López C212\\
    Daniel Alejandro Cárdenas C213\\
    }
\date{}

\begin{document}
    \maketitle

    \section{Demuestre el siguiente Teorema de existencia y Unicidad:\\
        Sea $Q = \left\{ (x,y) | a < x < b, c < y < d \right\}$ y las funciones $f_{1}(x),f_{2}(y)$ definidas y continuas en Q, 
        de modo que $f_2 \not= 0, \forall y \in (c,d)$. Entonces, por cada punto $ (x_0, y_0) \in Q $ pasa una y solo una curva 
        integral de la ecuación $\frac{\delta y}{\delta x} = f_{1}(x)f_{2}(y)$}

        \[\frac{\delta y}{\delta x} = f_1(x)f_2(y) | dividimos por f_2(y), f_2(y) \not = 0  \]
        \[\frac{1}{f_2(y)}\frac{\delta y}{\delta x} = f_1(x)\]

        \[\int \frac{1}{f_2(x)} \delta y = \int f_1(x) \delta x\]
        
        Como $f_2(y)$ es continuam $\frac{1}{f_2(y)}$ es continua
        
        Como $\frac{1}{f_2(y)}$ y $f_1(x)$ son continuas, son también integrables

        Como $\frac{1}{f_2(y)}$ y $f_1(x)$ son integrables, existe $F_2(y), F_1(x)$ primitivas, 
        tal que $ \int \frac{1}{f_2(y)} dy = F_2(y) + c_1$ y $ \int f_1(x) dx =  F_1(x) +c_2$ 

        Luego

        \[\int \frac{1}{f_2(y)} dy = \int f_1(x) dx \]
        \[F_2(y) + c_1 = F_1(x) + c_2\]
        \[c_1 - c_2 = F_1(x) - F_2(y) \]
        \[ C = F_1(x) - F_2(y) \]
        Obteniendo así la solución general

        Sabemos que pasa por el punto $(x_0, y_0)$ por lo tanto podemos particularizar la 
        solución evaluando en $F_1$ y $F_2$

        \[C = F_1(x_0)-F_2(y_0)\]

        Luego, $F_1(x_0)$ y $F_2(y_0)$ son únicos porque la evaluación de una función en un valor fijo es único

        Luego C es único por ser resta de valores únicos

        Luego, Queda demostrado que por cada punto $ (x_0, y_0) \in Q $ pasa una y solo una curva 
        integral de la ecuación $\frac{\delta y}{\delta x} = f_{1}(x)f_{2}(y)$
    \section{Ejercicio 21 página 81}

        \subsection{Datos de Interés}
        \begin{enumerate}
            \item Máxima capacidad del tanque: 500 gal
            \item Tanque lleno de Agua Pura $\Rightarrow$ x(0) = 0 y c(0) = 0
            \item Concentración de sal de entrada: 2 lb/gal
            \item Velocidad de Entrada de Salmuera: 5 gal/min
            \item Concentración de sal de salida: 2 lb/gal
            \item Velocidad de salida de Salmuera: 5 gal/min
            \item Tasa de entrada de sal: (2 lb/gal)(5 gal/min) 10 lb/min
            \item Tasa de salida de sal: ($\frac{x}{500}$ lb/gal)(5 gal/min) $\frac{x}{100}$ lb/gal  
        \end{enumerate}

        Planteemos el ejercicio como un problema de valor inicial

        \begin{equation}
            \frac{\delta x}{\delta t} + \frac{x}{100} = 10; x(0) = 0
        \end{equation}

        Sabemos que (1) es una ecuación diferencial linela, y su factor de integración es: $e^{\frac{t}{100}}$, luego:

        \begin{equation}
            \frac{\delta e^{\frac{t}{100}}}{\delta t} = 10 e^{\frac{t}{100}}
        \end{equation}

        integrando y resolviendo para x tenemos:

        \[ x(t) = 1000 + c e^{-\frac{t}{100}} ; x(0) = 0\]
        \[0 = 1000+c\]
        \[c = 1000\]

        \begin{center}
            \framebox{$x(t) = 1000 - 1000 e^{-\frac{t}{100}}$}
        \end{center}
    
    \section{Ejercicio 22 página 81}
        
        \subsection{¿Cuál es la concentración de sal $c(t)$ en el tanque en el tiempo t?}
            De la fórmula de la concentración $c = \frac{Ms}{Vd}$ donde Ms representa la masa 
            de soluto y Vd el volúmen de la disolución.\\ 
            Definimos la concenttración de Sal en el tiempo t como:
            \begin{equation}
                c(t) = \frac{x(t)}{500}
            \end{equation}
        \subsection{¿En t=5 min?}
            Sustituyendo en (3)
            \[ c(5) = \frac{x(5)}{500} \]
            \[ c(5) = \frac{1000-1000e^{-\frac{5}{100}}}{500} \]
            \[ c(5) = \frac{1000(1-e^{-\frac{1}{20}})}{500} \]
            \begin{center}
                \framebox{$c(5) \approx 48.77058$}
            \end{center}
    
        \subsection{¿Cuál es
        la concentración de sal en el tanque después de un largo
        tiempo, es decir cuando t → q ?}

            \[ \lim_{t \to \infty} x(t) * \frac{1}{500} \]
            \[ \lim_{t \to \infty} \frac{1000-1000e^{-\frac{5}{100}}}{500} \]

            Cuando $t \to \infty$, $e^{-\frac{5}{100}} = 0$, luego:

            \[ \lim_{t \to \infty} x(t) * \frac{1}{500} = \frac{1000}{500} = 2 \]
                
            Luego, la concentración pasado un largo período será de 2 lb/gal
        \subsection{¿En qué tiempo la concentración de sal en el tanque es igual a la mitad de este
        valor límite?}

        \[  \frac{x(t)}{500} = 1\]
        \[ x(t) = 500\]
        \[ 1000-1000e^{-\frac{5}{100}} = 500 \]
        \[ -1000e^{-\frac{5}{100}} = -500 \]
        \[ 500e^{-\frac{5}{100}} = 1 \]
        \[  e^{-\frac{5}{100}} = \frac{1}{500} \]
        \[ \frac{t}{100} = \ln(500) \]
        \begin{center}
            \framebox{$t = 100\ln(500)$}
        \end{center}

\end{document}